\section{Localization}\label{sec:localization}

One of the major components of any navigation system is the localization subsystem. The job of the localization subsystem on any robot is to determine where the robot is. Knowing where the robot is essential to being able to actually navigate to a specific goal position, otherwise the robot won't know once it reaches that goal position. The pose estimates generated by the localization subsystem may also be used in control loops or otherwise by the planning and control subsystems. For example, the local planner described in \autoref{subsec:base_local_planner} uses the pose (2D position and heading) as well as the translational and rotational velocities of the base when generating possible trajectories to evaluate. The precision navigation system described in this thesis uses the pose and velocity estimates as control variables to generate commands that follow the desired path \todo{reference wherever I eventually describe the steering algorithms}.

For precise navigation, the robot not only needs to know where it is, but that pose esimate must be at least as accurate as the precision navigation system as a whole is designed to be. The localization subsystem used by the precision navigation system described in this thesis is broken up into two separate components, depending on what type of reference frame their pose estimates are in: the relative localization component and the absolute localization component.

\subsection{Relative Localization}\label{subsec:relative_localization}

The first major component of the localization subsystem used in this thesis is the relative frame localization system. This component generates pose estimates relative to wherever the robot was powered on. The relative localization subsystem also generates estimates of the translation and rotational velocities. While these estimates are important, the relative frame position estimate is only good for uses that can tolerate a reference frame that drifts over long periods of time and is not fixed between times when the robot is powered on and off. For these reasons, it is not useful for describing goal points or global planning. It is useful for local planning and collision avoidance, as both of those are tolerant to drift in the reference frame. The estimates generated by the relative frame localization are also useful for the control algorithms used in this thesis because they can be generated at a high rate thanks to the computional simplicity of the algorithms used compared to absolute frame localization algorithms as described in \autoref{subsec:absolute_localization}.

HARLIE's relative frame localization is generated by using an Extended Kalman Filter (EKF) \autocites{Larsen1999}{Welch95anintroduction}{ProbRobotics}. The EKF is an extension of the standard Kalman Filter algorithm, used for optimal state estimation in linear systems, to non-linear systems such as a differential drive robot. Using a model of the dynamic system and measurements from a variety of sensors, the EKF is able to produce a state estimate that is more accurate than any sensor individually. The EKF on HARLIE uses two of the sensors described in \autoref{subsec:harlie_setup} to produce these relative frame state estimates: the encoders and the gyroscope. \autoref{alg:harlie_ekf_algorithm} is a description of the exact EKF algorithm used on HARLIE. The values of each set of matrices and functions used are detailed afterwards.

\begin{algorithm}
\caption{HARLIE's EKF Relative Localization Algorithm}
\label{alg:harlie_ekf_algorithm}
\DontPrintSemicolon
\SetKwFunction{ZeroOutBiasXYThetaCovariance}{ZeroOutBiasXYThetaCovariance}

\KwIn{$x_{t-1}$, $P_{t-1}$, $z_t$}
\KwOut{$x_t$, $P_t$}

$\hat{x_t} = f(x_{t-1})$\;
$\hat{P_t} = G \cdot P_{t-1} \cdot G^T + Q$\;
$\hat{P_t} = $ \ZeroOutBiasXYThetaCovariance{$\hat{P_t}$}\;
\ForEach{$z^i_t \in z_t$}
{
	$y^i_t = z^i_t - h(\hat{x_t})$\;
	$S = H \cdot \hat{P_t} \cdot H^T + R$\;
	$K = \hat{P_t} \cdot H^T \cdot S^{-1}$\;
	$\hat{x_t} = \hat{x_t} + K \cdot y^i_t$\;
	$\hat{P_t} = \left(I - K \cdot H \right) \cdot \hat{P_t}$\;
}
$x_t = \hat{x_t}$\;
$P_t = \hat{P_t}$\;
\end{algorithm}

\subsection{Absolute Localization}\label{subsec:absolute_localization}

\begin{comment}
This section details the PSO used on HARLIE and has data and figures and shit for how accurate it is

Outline:
	Why do we need localization? Why do we need PRECISE localization?

	Parts
		Relative Localization
			EKF on HARLIE
		Absolute Localization
			AMCL algorithm
			Needs tuned to prevent pops
	Results

\end{comment}
