\section{Introduction}\label{sec:introduction}

Mobile robotics is a growing area in the field of robotics. The key difference between mobile robotics and other types of robotics, such as industrial automation, is that mobile robots can navigate their environment. While there have been mobile robots such as tour guides or self-driving cars for years now, new classes of mobile robots such as service robots or smart wheelchairs are pushing the limits of current navigation technologies.

While there are mature technologies for outdoor navigation, as is the case with field robotics or self-driving cars, as well as simply moving from point A to point B indoors while avoiding obstacles, as is the case with tour guide robots, there is relatively little available for doing precision navigation indoors or outdoors. Whereas traditional navigation approaches often include path followers that simply make a ``best effort'' to precisely follow the overall path plan while avoiding obstacles, a precision navigation system is designed to precisely follow the overall path plan at all times while still avoiding obstacles.

Precision navigation allows mobile robots to function in situations that are outside the capabilities of standard robot navigation techniques. For example, with precision guarantees, a smart wheelchair can smoothly and reliably pass through doorways with little margin for error or pull up parallel and close enough to a wall so that the user can press a handicap door assist button. Standard navigation techniques do not make strong guarantees about achieving goal positions via a precise path and it would be very difficult for many navigation systems to approach a wall so closely without making many attempts. Even though a precision navigation system must always precisely follow the path, it must also deal with dynamic obstacles that appear in the path. Because a precision navigation system does not allow the robot to intentionally deviate from the planned path, the planned path must be updated via dynamic replanning in order to account for and avoid any dynamic obstacles.

The precision navigation system described in this thesis is designed to do all of those things and such behaviors were experimentally confirmed both in simulation and with a physical robot, Case Western Reserve University's HARLIE. The remainder of this thesis is organized as follows. \autoref{sec:experimental_system} describes the experimental systems used, HARLIE and the Gazebo simulation environment, to test the precision navigation system. \autoref{sec:related_work} describes related navigation systems, specifically open source navigation systems that were evaluated on HARLIE. \autoref{sec:localization} describes the precision localization subsystem used by this precision navigation system to determine the robot's current pose. \autoref{sec:steering} describes the steering component of this precision navigation system, which outputs driving commands to the robot. \autoref{sec:trajectory_generation} describes the trajectory generation component of this precision navigation system, which takes a planned path and outputs the next state the robot needs to achieve in order to follow that path. \autoref{sec:path_planning} describes the simplistic path planner developed for this thesis. \autoref{sec:results} describes a number of experimental results for the different components of the precision navigation system. \autoref{sec:future_work} describes possible avenues for extending the precision navigation system described in this thesis.
