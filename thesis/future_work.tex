\section{Future Work}\label{sec:future_work}

There are a number of different avenues for building upon the precision navigation system described in this thesis. Some are described in more detail in this section.

One area for future work is in extensions to the precision localization subsystem described in \autoref{sec:localization}. While the precision localization system was highly effective indoors with the very accurate laser scanner, \emph{a priori} map and accurate relative localization system, there are cases where some of these components are unavailable. For example, on a robot that needs to navigate precisely in new buildings, an \emph{a priori} map would not be readily available. SLAM algorithms could provide localization within unknown environments, but whether those localization estimates would be precise enough for the precision navigation system is unknown and would be an avenue for future work. Similarly, some robots, such as Otto, the Smart Wheelchair developed by Chad Rockey in \autocite{Rockey2012}, do not possess highly accurate laser scanners -- analysis of how to utilize different, less accurate sensors for precision localization would be another area for future work. Currently, the precision localization system is limited to indoor environments only; more work would be required to extend the precision localization system to work in an outdoors environment.

Another area for future work would be more work characterizing the performance of the steering algorithms as well as trying new steering algorithms. A detailed analysis of the steering algorithms would extend the results presented in \autoref{subsec:phase_space_steering_skills}. Such an analysis could include things such as derivation and testing of equations that would allow a user to predict tangential distance to path convergence or tangential distance before overshoot, given the robot's initial conditions and gains. With the ability to predict those results for different gain values, it would become much easier to tune the steering algorithms to the desired behavior for a particular environment. 

A different extension to the steering and trajectory generation that could be useful would be changing the path segment description from the simple geometric parameterization currently in use to a spline-based representation. Splines would allow each path segment to be more complex than a simple straight line, constant curvature arc or spin-in-place; splines could also be easier to fit to a sequence of points, such as those output by common planning algorithms or points learned by driving a desired path to ``teach'' the robot, than the current path segment representation.

Future work could also be devoted towards improving the simplistic path planner presented in \autoref{sec:path_planning}. The planner presented there was rather limited, especially for use in a new environment where there are no \emph{a priori} paths. Such a path planner could be built on some of the different dynamic planning algorithms discussed as alternatives to navfn in \autoref{subsec:navfn}. Such an extension was explored by Bill Kulp in \autocite{Kulp2012}, utilizing the SBPL library \autocite{Likhachev2010} to generate paths that were converted to the format described in \autoref{subsec:path_segment}.

As part of this thesis, numerous avenues for integration with a ``Smart'' Building were discovered. A Smart Building is one that would be able to send information such as maps, goal locations with annotations and paths to the precision navigation system. One area where Smart Building integration would be important is in the specification of complex paths that would be very difficult for any autonomous path planner to determine. One example path that would be a good candidate for including in a Smart Building is the sequence of steps required for a power wheelchair user to enter the Case Western Reserve University Glennan building after hours or on weekends. In order to get into the building, a power wheelchair user must first drive up a handicap access ramp, then drive up to a card reader that unlocks the door with an ID card, then back up to the handicap power assist button to open the door, and then wait until the door opens before attempting to proceed through. These steps are further complicated by the fact that, after getting to the top of the handicap access ramp and driving to the card reader or door, there is a set of stairs that the wheelchair user could fall down if they steer slightly off to one side. Getting this sequence correct would be tricky for a path planner with no \emph{a priori} knowledge of the specific steps required for the Glennan building; the sequence is trivial if pre-planned and the building could download the pre-planned paths to an approaching robotic wheelchair.

\begin{comment}

Details possible areas to expand on this thesis and improve it's performance on our wheelchair platform

Talk about things where a Smart Building can \emph{definitely} help the precision navigation out. Some are obvious such as goal annotation so that things like ``kitchen'' mean something, but others are less obvious.

One of those would be things that would be tricky, even for a good path planner, such as the approach to get into Glennan from the quad-level.

\end{comment}