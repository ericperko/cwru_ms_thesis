\section{Conclusion}\label{sec:conclusion}

This thesis has presented a precision navigation system for use on an indoor mobile robot. The precision navigation system was developed to address shortcomings in the ROS navigation stack when used for precision navigation on a differential drive robot. While some ROS packages, such as costmap\_2d, base\_local\_planner and navfn, were replaced by components developed in this thesis, other ROS components such as the core infrastructure (available in both C++ and Python), visualization tools, message generation, amcl and gmapping packages were used without modification. These components accelerated development of the precision navigation system described in this thesis.

Using a planar laser scanner, wheel encoders, a gyroscope and an \emph{a priori} map, a precision localization system was developed in order to generate quick, precise pose estimates for the mobile robot. Using these precise pose estimates, a precision path execution system made up of a steering algorithm, a trajectory generator and a simplistic path planner was developed. This system allowed HARLIE to precisely navigate indoors, including passing through doorways with less than seven centimeters of clearance on either side or pulling up parallel to a door reader. While the path planner was simplistic, it included the ability to splice in a fixed sequence of path segments to avoid unexpected obstacles.

The precision navigation system presented in this thesis also has a number of opportunities for further work. Some possibilities include allowing the precision localization system to operate without an \emph{a priori} map or in an outdoor environment, a more rigorous analysis of the properties of the steering algorithms, extending the path segment description to include splines or improving the simplistic path planner. Some areas of possible integration with a ``Smart'' Building were also discussed, such as providing maps or complex pre-planned paths to a mobile robot entering the building for the first time.